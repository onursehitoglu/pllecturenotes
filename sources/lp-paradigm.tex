%\documentclass[compress,dvips,xcolor=table]{beamer}
\usepackage{etex}
%\documentclass{article}
%\usepackage{beamerarticle}
\usepackage{pstricks,pst-node} % PSTricks package
%\usepackage[turkish]{babel}
\usepackage[utf8]{inputenc}
\usepackage{listings}
\usepackage{multicol}
%\includeonlyframes{current}

\def\circtxt#1{$\mathalpha \bigcirc \mkern-13mu \mathtt #1$}

\mode<article>
{
  \usepackage{fullpage}
  \usepackage{pgf}
  \usepackage{hyperref}
}

\mode<presentation>
{
  \usetheme{metuceng}

%  \setbeamercovered{transparent}
}


\title{Programming Language Concepts}
\subtitle{Logic Prog. Paradigm}
\author{Onur Tolga Şehitoğlu}
\institute[ODTÜ]{Bilgisayar Mühendisliği}
\subject{Logic Programming}
\date{}
	\titlegraphic{\insertmetutitle\insertlicense}


\begin{document}
\lstset{language=C,
        basicstyle=\scriptsize\ttfamily,
        keywordstyle=\color{blue!50!black}\bfseries,
        identifierstyle=\color{blue!60!green}\sffamily,
        stringstyle=\color{red!70!green}\ttfamily,
	commentstyle=\color{blue!30!white}\itshape,
        showstringspaces=true}
\setbeamercolor{hexample}{bg=green!5!white,fg=black}%
\setbeamercolor{cexample}{bg=blue!5!white,fg=black}%
\setbeamercolor{pexample}{bg=orange!5!white,fg=black}%
\setbeamercolor{oexample}{bg=violet!5!white,fg=black}%

 \frame[plain]{\maketitle}
 \begin{frame}
 \frametitle{Outline}
 \begin{multicols}{2}
 \tableofcontents
 \end{multicols}
 \end{frame}

\section{Logic Programming Paradigm}
\begin{frame}
\frametitle{Logic Programming Paradigm}
\begin{itemize}
\item Based on logic and \structure{declarative programming}
\item 60's and early 70's 
\item Prolog (\textbf{Pro}gramming in \textbf{log}ic, 1972) is
	the most well known representative of the paradigm.
\item Prolog is based on \structure{Horn clauses} and 
	\structure{SLD resolution} 
\item Mostly developed in \structure{fifth generation computer systems
	project}
\item Specially designed for theorem proof and artificial intelligence but
	allows general purpose computation.
\item Some other languages in paradigm: ALF, Frill, Gödel, Mercury, Oz,
Ciao, $\lambda$Prolog, datalog, and CLP languages
\end{itemize}
\end{frame}

\begin{frame}
\frametitle{Constraint Logic Programming}
\begin{itemize}
\item
Clause: disjunction of universally quantified literals,
\[ \forall(L_1 \vee L_2 \vee ... \vee L_n) \]
\item
A logic program clause is a clause with exactly one
positive literal
\[ \begin{array}{l}
   \forall(A \vee \neg A_1 \vee \neg A_2 ... \vee \neg A_n) \equiv \\
   \forall(A \Leftarrow A_1 \wedge A_2 ... \wedge  A_n) 
\end{array}\] 

\item
A goal clause: no positive literal
\[ \forall(\neg A_1 \vee \neg A_2 ... \vee \neg A_n) \]

\item
Proof by refutation, try to unsatisfy the clauses with
a goal clause $G$. Find $\exists(G)$.
\item
Linear resolution for definite programs with constraints and
selected atom.
\end{itemize}
\end{frame}

\defverbatim[colored]\codefamily{
\begin{lstlisting}[language=Prolog,escapechar=\#]
father(ahmet,ayse).
father(hasan,ahmet).
mother(fatma,ayse).
mother(hatice,fatma).
parent(X,Y) :- father(X,Y).
parent(X,Y) :- mother(X,Y).
grandparent(X,Y) :- parent(X,Z),parent(Z,Y).
\end{lstlisting}}
\begin{frame}
\frametitle{What does Prolog look like?}
\begin{beamercolorbox}{oexample}
\codefamily
\end{beamercolorbox}

\end{frame}

\section{Prolog basics}
\begin{frame}
\begin{itemize}
\item CLP on first order terms. (Horn clauses).
\item \structure{Unification}. Bidirectional.
\item \structure{Backtracking}. Proof search based on trial of all matching
	clauses.
\end{itemize}
\end{frame}

\section{Prolog Terms}
\begin{frame}
\frametitle{Prolog Terms}
\begin{itemize}
\item Atoms:
	\begin{enumerate}
	\item Strings with starting with a small letter and consist of
		\lstinline![a-zA-Z_0-9]*!
	\item Strings consisting of only punctuation
	\item Any string enclosed in back quotes
	\end{enumerate}
\item Numbers
\item Variables:
	\begin{enumerate}
	\item Strings with starting with a capital letter or {\tt \_} and consist of
		\lstinline![a-zA-Z_0-9]*!
	\item {\tt \_} is the universal match symbol. Not variable
	\end{enumerate}
\end{itemize}
\end{frame}
\begin{frame}
\begin{itemize}
\item Structures:
	\begin{itemize}
	\item starts with an atom head
	\item has one or more arguments encolsed in paranthesis, separated by comma
	\item structure head cannot be a variable or anything other than atom.
	\end{itemize}
\end{itemize}
\end{frame}
\begin{frame}
\begin{itemize}
\item\ 
\end{itemize}
\end{frame}
\end{document}
